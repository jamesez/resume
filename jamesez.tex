\documentclass[letterpaper,10pt]{article}

\usepackage{palatino}
\usepackage{latexsym}
\usepackage[empty]{fullpage}
\usepackage{titlesec}
\usepackage[usenames,dvipsnames]{color}
\usepackage{verbatim}
\usepackage{enumitem}
\usepackage[hidelinks]{hyperref}
\usepackage{fancyhdr}
\usepackage[english]{babel}
\usepackage{tabularx}
\input{glyphtounicode}

\pagestyle{fancy}
\fancyhf{} % clear all header and footer fields
\fancyfoot[C]{\thepage{}}
\renewcommand{\headrulewidth}{0pt}
\renewcommand{\footrulewidth}{0pt}

% Adjust margins
\addtolength{\oddsidemargin}{-0.5in}
\addtolength{\evensidemargin}{-0.5in}
\addtolength{\textwidth}{1in}
\addtolength{\topmargin}{-.5in}
\addtolength{\textheight}{1.0in}

\urlstyle{same}

\raggedbottom
\raggedright
\setlength{\tabcolsep}{0in}

% Sections formatting
\titleformat{\section}{
  \vspace{-4pt}\scshape\raggedright\large
}{}{0em}{}[\color{black}\titlerule \vspace{-5pt}]

% Ensure that generate pdf is machine readable/ATS parsable
\pdfgentounicode=1

%-------------------------
% Custom commands
\newcommand{\resumeItem}[2]{
  \item\small{
    \textbf{#1}{\hspace{0.5em}#2}
  }
}

% Just in case someone needs a heading that does not need to be in a list
\newcommand{\resumeHeading}[4]{
    \begin{tabular*}{0.99\textwidth}[t]{l@{\extracolsep{\fill}}r}
      \textbf{#1} & #2 \\
      \textit{\small#3} & \textit{\small #4} \\
    \end{tabular*}\vspace{-5pt}
}

\newcommand{\resumeSubheading}[4]{
  \vspace{-1pt}\item
    \begin{tabular*}{0.97\textwidth}[t]{l@{\extracolsep{\fill}}r}
      \textbf{#1} & #2 \\
      {\small #3} & {\small #4} \\
    \end{tabular*}\vspace{-5pt}
}

\newcommand{\resumeSubSubheading}[2]{
    \begin{tabular*}{0.97\textwidth}[t]{l@{\extracolsep{\fill}}r}
      \textit{\small #1} & \textit{\small #2} \\
    \end{tabular*}\vspace{-5pt}
}

\newcommand{\resumeSubItem}[2]{\resumeItem{#1}{#2}\vspace{-4pt}}

\renewcommand{\labelitemii}{$\circ$}

\newcommand{\resumeSubHeadingListStart}{\begin{itemize}[leftmargin=*]}
\newcommand{\resumeSubHeadingListEnd}{\end{itemize}}
\newcommand{\resumeItemListStart}{\begin{itemize}}
\newcommand{\resumeItemListEnd}{\end{itemize}\vspace{-5pt}}

%-------------------------------------------
%%%%%%  CV STARTS HERE  %%%%%%%%%%%%%%%%%%%%%%%%%%%%


\begin{document}
\thispagestyle{empty}
%----------HEADING-----------------
\begin{tabular*}{\textwidth}{l@{\extracolsep{\fill}}r}
  \textbf{\href{http://jamesez.io/}{\Large Jim Zajkowski}} & \href{mailto:jim@jamesez.io}{jim@jamesez.io}\\
  \href{http://jamesez.io/}{http://jamesez.io} & 734-846-1988 \\
\end{tabular*}


%-----------EXPERIENCE-----------------
\section{Experience}
  \resumeSubHeadingListStart

    \resumeSubheading
      {Information Technology Services, University of Michigan}{Ann Arbor, MI}
      {MiWorkspace Engineering}{}

      \resumeSubSubheading
        {DevOps Architect Supervisor}{2017--Present}
        \resumeItemListStart
          \resumeItem{Tensorflow}
            {TensorFlow is an open source software library for numerical computation using data flow graphs; primarily used for training deep learning models. Worked on APIs and performance for training models on Tensor Processing Units (TPU).}
          \resumeItem{Apache Beam}
            {Apache Beam is a unified model for defining both batch and streaming data-parallel processing pipelines, as well as a set of language-specific SDKs for constructing pipelines and runners.}
        \resumeItemListEnd

      \resumeSubSubheading
        {Mac Team Lead}{2015--2017}
        \resumeItemListStart
          \resumeItem{Apache Beam}
            {Apache Beam is a unified model for defining both batch and streaming data-parallel processing pipelines}
        \resumeItemListEnd

      \resumeSubSubheading
        {Platform Senior Engineer}{2012--2015}
        \resumeItemListStart
          \resumeItem{Apache Beam}
            {Apache Beam is a unified model for defining both batch and streaming data-parallel processing pipelines}
        \resumeItemListEnd

%-------------------------------------------

    \resumeSubheading
      {College of Literature, Sciences \& Arts, University of Michigan}{Ann Arbor, MI}
      {LSA Mac Lead}{2010--2013}
      \resumeItemListStart
        \resumeItem{Simple NetInstall}
          {Managed the College's Mac deployment system, updating it for Mac OS X 10.6 and new packages every semester. This system was end-of-lifed with 10.7.}
        \resumeItem{Izzy}
          {Using munki, Rails, and packagemaker, I created a lifecycle management system for Mac OS at the college, Izzy. Izzy let the support staff in the field deploy new packages to systems at any time though a simple web interface, rather than at build time with NetInstall or by hand, saving hundreds of hours. Izzy additionally let units create templates for their builds, and each department's access was siloed. Izzy immediately ramped to over 1,000 systems in five months.}
        \resumeItem{Package Automation}
          {I automated some very common packages (Firefox, Adobe Flash Player, Chrome, etc) using Jenkins. This allowed the college to be extremely timely with Flash security updates, deploying updates within hours of the new version being released, and requiring no manual steps.}
        \resumeItem{IzzyBoot}
          {Created a bare-metal imaging workflow for Macs, by creating an Objective-C wrapper around asr. IzzyBoot leveraged a decision engine in Izzy to pick which base OS was appropriate for the device, as Mac OS versions were occasionally model-specific. The wrapper would register the client with Izzy as well. This tool made building (and re-building) systems faster and more reliable than manually installing the OS, as the Mac would reboot into munki's package installation after imaging.}
      \resumeItemListEnd

    \resumeSubheading
      {Life Sciences Institute, University of Michigan}{Ann Arbor, MI}
      {System Administrator}{2003--2010}
      \resumeItemListStart
        \resumeItem{Infrastructure}
          {Designed and implemented entire computing infrastructure of a new institute from the ground up, including all-new switchgear, multiple Xserve RAIDs in a SAN, an Xserve-based compute cluster, Open Directory on Mac OS X Server, VMware vSphere, Active Directory, SCCM, and Exchange.}
        \resumeItem{Desktop Engineering}
          {Created a Mac OS build system based on NetInstall and some scripts, making builds repeatable, easy to manage, and simple for student staff to perform. Implemented a ``golden-triangle'' configuration with Active Directory and Open Directory. Became a Certified Mac Technician and formed an AppleCare SSO, speeding up our turn-around time for repairs. Provided repair services to nearby University units. Implemented SCCM for Windows desktops.}
        \resumeItem{Purchasing System}
          {Created an institute-wide purchasing app to streamline and centralize ordering, reducing inefficiencies and allowing researchers to focus on science.}
      \resumeItemListEnd

    \resumeSubheading
      {IT Contract Services, University of Michigan}{Ann Arbor, MI}
      {System Administrator}{2000--2003}
      \resumeItemListStart
        \item\small{
          Supported Solaris and RedHat Linux customers across campus.
        }
      \resumeItemListEnd

    \resumeSubheading
      {Surveys of Consumers, University of Michigan}{Ann Arbor, MI}
      {Web Developer}{1996--Present}
      \resumeItemListStart
        \item\small{
          Develop and manage the Surveys web properties, including internal-, public- and sponsor-facing data distribution sites and tools. Primarily implemented in PHP and Ruby on Rails.
        }
      \resumeItemListEnd

  \resumeSubHeadingListEnd

%-----------EDUCATION-----------------
\section{Education}
\resumeSubHeadingListStart
  \resumeSubheading
    {University of Michigan}{Ann Arbor, MI}
    {Bachelor of Science in Engineering, Computer Engineering \em{with honors}}{1996--2000}
\resumeSubHeadingListEnd


%-----------PROJECTS-----------------
\section{Projects}
  \resumeSubHeadingListStart
    \resumeSubItem{QuantSoftware Toolkit}
      {Open source python library for financial data analysis and machine learning for finance.}
    \resumeSubItem{Github Visualization}
      {Data Visualization of Git Log data using D3 to analyze project trends over time.}
    \resumeSubItem{Recommendation System}
      {Music and Movie recommender systems using collaborative filtering on public datasets.}
    \resumeSubItem{Mac Setup}
      {Book that gives step by step instructions on setting up developer environment on Mac OS.}
  \resumeSubHeadingListEnd

%
%--------PROGRAMMING SKILLS------------
%\section{Programming Skills}
%  \resumeSubHeadingListStart
%    \item{
%      \textbf{Languages}{: Scala, Python, Javascript, C++, SQL, Java}
%      \hfill
%      \textbf{Technologies}{: AWS, Play, React, Kafka, GCE}
%    }
%  \resumeSubHeadingListEnd


%-------------------------------------------
\end{document}
